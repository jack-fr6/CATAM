\documentclass{article}
\usepackage{float}
\usepackage{enumitem}
\usepackage{hyperref}
\usepackage[framed, numbered]{matlab-prettifier}
\usepackage[margin=0.5in]{geometry}
\usepackage[framed, numbered]{matlab-prettifier}
\usepackage{amsfonts}
\usepackage{amsmath}
\usepackage{multirow}
\hypersetup{
    colorlinks,
    citecolor=black,
    filecolor=black,
    linkcolor=black,
    urlcolor=black
}
\usepackage{graphicx}
\graphicspath{ {./Images/} }
\begin{document}
\title{Galois Groups}
\maketitle
\tableofcontents
\newpage
\section*{Preface}
This is my CATAM project, 16.1 for part II. The code for each question can be found in section $4$.
\newpage
\section{Question 1}
See the table below for some outputs for my programs which calculates the quotient $q$ remainder $r$ when dividing polynomials $f$ by $g$ over $\mathbb{F}_p$.

\begin{table}[hp]
\begin{center}
\begin{tabular}{|l|l|l|l|l|}
\hline
f                        & g                    & q                      & r                       & p  \\ \hline
$x^5 - 11x^3 + 22x - 11$ & $x^4 - x^3 - 4x + 16$ & $x + 1$                  & $4x^3 + 4x^2 + 3x + 1$ & 7  \\
$x^5 - 11x^3 + 22x - 12 $& $x^3 + x^2 - 2x - 1$  & $x^2 + 4x + 2$          & $2x^2 + 1$               & 5  \\
$x^4 - x^3 - 4x + 16   $  & $x+1                $  & $x^3 + 11x^2 + 2x + 7$ & $9$                       & 13 \\
$x^3+3x+1 $                & $x^2+3               $ & $x$                      & $1       $                & 2  \\ \hline
\end{tabular}
\caption{ Various outputs of polynomial division program }
\end{center}
\end{table}

See the table below for some outputs for my programs which calculates the GCD of two polynomials $f,g$ over $\mathbb{F}_p$.


\begin{table}[hp]
\begin{center}
\begin{tabular}{|l|l|l|l|}
\hline
f               & g                & GCD        & p  \\ \hline
$x^2+x+1$       & $x+1$            & 1          & 3  \\
$x^2+2x+1$      & $x^3+3x^2+3x+1$  & $x^2+2x+1$ & 17 \\
$x^6+5x^3+6x+1$ & $x^5+3x^3+12x+7$ & $3x+4$     & 11 \\
$x^6+1$         & $x^3+1$          & 2          & 23 \\ \hline
\end{tabular}
\caption{ Various outputs of polynomial GCD program }
\end{center}
\end{table}



A way to efficiently calculate the power,  say $n$, of a polynomial $f$ modulo a polynomial $g$ using my programs written in this question is as follows:
	\begin{enumerate}
		\item Firstly write $n$ in base $2$,  say $n=b_n\ldots b_0=\sum_{i=0}^nb_i2^{i}$. 
		\item Iteratively calculate $f^{2^i}\pmod g$ via $f^{2^i}=(f^{2^{i-1}})^2$.  Then use the polynomial division algorithm to reduce it modulo $g$.
		\item Iteratively calculate $\prod_{i=0}^kf^{b_i2^i}\pmod g$ by consecutively multiplying it by $f^{2^{k+1}}$  if $b_{k+1}=1$ and $1$ otherwise.  At each stage use the polynomial division program to reduce the expression modulo $g$.
	\end{enumerate} 

\section{Question 3}
Find below the tables of the decomposition group of each polynomial for each prime between $2$ and $97$.



\begin{table}[hp]
\begin{center}
\begin{tabular}{|c|lllllllllllll|}
\hline
\multirow{2}{*}{Polynomial}        & \multicolumn{13}{c|}{Prime}                                                 \\ \cline{2-14} 
                                   & 2   & 3   & 5   & 7   & 11  & 13  & 17  & 19  & 23  & 29  & 31  & 37  & 41  \\ \hline
$x^2+x+41$                         & $C_2$ & $C_2$ & $C_2$ & $C_2$ & $C_2$ & $C_2$ & $C_2$ & $C_2$ & $C_2$ & $C_2$ & $C_2$ & $C_2$ & $C_1$ \\
$x^3+2x+1$                         & $C_2$ & $C_3$ & $C_3$ & $C_3$ & $C_2$ & $C_2$ & $C_1$ & $C_3$ & $C_2$ & $C_3$ & $C_2$ & $C_2$ & $C_3$ \\
$x^3+x^2-2x-1$                     & $C_3$ & $C_3$ & $C_3$ & -   & $C_3$ & $C_1$ & $C_3$ & $C_3$ & $C_3$ & $C_1$ & $C_3$ & $C_3$ & $C_1$ \\
$x^4-2x^2+4$                       & -   & -   & $C_2$ & $C_2$ & $C_2$ & $C_2$ & $C_2$ & $C_1$ & $C_2$ & $C_2$ & $C_2$ & $C_2$ & $C_2$ \\
$x^4-x^3-4x+16$                    & -   & -   & $C_4$ & $C_4$ & -   & $C_2$ & $C_2$ & $C_4$ & $C_2$ & $C_2$ & $C_2$ & $C_2$ & $C_2$ \\
$ x^4 - 2x^3 + 5x + 5$             & $C_4$ & -   & -   & $C_4$ & $C_4$ & $C_3$ & $C_3$ & $C_3$ & $C_4$ & $C_3$ & $C_3$ & $C_4$ & $C_3$ \\
$x^4 + 7x^2 + 6x + 7$              & -   & -   & $C_2$ & $C_1$ & $C_2$ & -   & $C_2$ & $C_1$ & $C_2$ & $C_2$ & $C_1$ & $C_1$ & $C_2$ \\
$ x^4 + 3x^3 - 6x^2 - 9x + 7$      & -   & $C_2$ & -   & $C_2$ & $C_2$ & $C_2$ & $C_2$ & $C_2$ & $C_2$ & $C_2$ & $C_1$ & $C_2$ & -   \\
$ x^5 + 36$                        & -   & -   & -   & $C_4$ & $C_5$ & $C_4$ & $C_4$ & $C_2$ & $C_4$ & $C_2$ & $C_1$ & $C_4$ & $C_5$ \\
$ x^5 - 5x + 3$                    & $ C_6$ 	& $C_2$ &	- &  -	& $C_3$	& $C_2$	&$ C_2$	&$ C_2$	&$ C_6$	& $C_3$	&$ C_2$	& $C_6$	&$ C_2$ \\
$ x^5 + x^3 - 3x^2 + 3$            & -   & -   & $C_5$ & $C_3$ & $C_3$ & $C_5$ & $C_5$ & $C_5$ & $C_5$ & $C_2$ & $C_2$ & $C_5$ & -   \\
$ x^5 - 11x^3 + 22x - 11 $         & $C_5$ & $C_5$ & $C_5$ & $C_5$ & -   & $C_5$ & $C_5$ & $C_5$ & $C_1$ & $C_5$ & $C_5$ & $C_5$ & $C_5$ \\
$ x^6 + x + 1 $                    & $C_6$ & $C_6$ & $C_3$ & $C_5$ & $C_6$ & $C_6$ & $C_6$ & $C_4$ & $C_3$ & $C_6$ & $C_6$ & $C_6$ & $C_4$ \\
$ x^7 - 2x^6 + 2x + 2 $            & -   & -   & $C_7$ & $C_7$ & -   & $C_6$ & $C_4$ & $C_4$ & $C_7$ & $C_7$ & $C_7$ & $C_7$ & $C_5$ \\
$ x^7 + x^4 - 2x^2 + 8x + 4 $      & -   & -   & $C_6$ & $C_6$ & $C_2$ & $C_2$ & $C_2$ & $C_3$ & $C_2$ & $C_6$ & $C_2$ & $C_2$ & $C_6$ \\
$ x^7 + x^5 - 4x^4 - x^3 + 5x + 1$ & $C_7$ & -   & $C_2$ & $C_7$ & $C_2$ & $C_2$ & $C_7$ & $C_7$ & $C_2$ & $C_7$ & $C_7$ & $C_7$ & $C_7$ \\ \hline
\end{tabular}
\caption{ Decomposition groups of various polynomials for primes between $2$ and $41$}
\end{center}
\end{table}
\newpage




\begin{table}[hp]
\begin{center}
\begin{tabular}{|c|llllllllllll|}
\hline
\multirow{2}{*}{Polynomial}        & \multicolumn{12}{c|}{Prime}                                           \\ \cline{2-13} 
                                   & 43  & 47  & 53  & 59  & 61  & 67  & 71  & 73  & 79  & 83  & 89  & 97  \\ \hline
$x^2+x+41$                         & $C_1$ & $C_1$ & $C_1$ & $C_2$ & $C_1$ & $C_2$ & $C_1$ & $C_2$ & $C_2$ & $C_1$ & $C_2$ & $C_1$ \\
$x^3+2x+1$                         & $C_2$ & $C_2$ & $C_3$ & -   & $C_2$ & $C_2$ & $C_1$ & $C_2$ & $C_3$ & $C_2$ & $C_2$ & $C_2$ \\
$x^3+x^2-2x-1$                     & $C_1$ & $C_3$ & $C_3$ & $C_3$ & $C_3$ & $C_3$ & $C_1$ & $C_3$ & $C_3$ & $C_1$ & $C_3$ & $C_1$ \\
$x^4-2x^2+4$                       & $C_1$ & $C_2$ & $C_2$ & $C_2$ & $C_2$ & $C_1$ & $C_2$ & $C_1$ & $C_2$ & $C_2$ & $C_2$ & $C_1$ \\
$x^4-x^3-4x+16$                    & $C_4$ & $C_2$ & $C_4$ & $C_2$ & $C_2$ & $C_2$ & $C_2$ & $C_2$ & $C_4$ & $C_2$ & $C_4$ & $C_1$ \\
$ x^4 - 2x^3 + 5x + 5$             & $C_2$ & $C_3$ & $C_2$ & $C_3$ & $C_4$ & $C_3$ & $C_3$ & $C_3$ & -   & $C_4$ & $C_4$ & $C_2$ \\
$x^4 + 7x^2 + 6x + 7$              & $C_1$ & $C_2$ & $C_2$ & $C_2$ & $C_1$ & $C_1$ & $C_2$ & $C_1$ & $C_1$ & $C_2$ & $C_2$ & $C_1$ \\
$ x^4 + 3x^3 - 6x^2 - 9x + 7$      & $C_2$ & $C_2$ & $C_2$ & $C_2$ & $C_2$ & $C_2$ & $C_1$ & $C_2$ & $C_1$ & $C_2$ & $C_1$ & $C_2$ \\
$ x^5 + 36$                        & $C_4$ & $C_4$ & $C_4$ & $C_2$ & $C_5$ & $C_4$ & $C_5$ & $C_4$ & $C_2$ & $C_4$ & $C_2$ & $C_4$ \\
$ x^5 - 5x + 3$                    & $C_2$ & $C_2$ & $C_2$ & $C_2$ & $C_6$ & $C_3$ & $C_2$ & $C_3$ & $C_2$ & $C_2$ & $C_4$ & $C_5$ \\
$ x^5 + x^3 - 3x^2 + 3$            & $C_3$ & $C_3$ & $C_5$ & $C_3$ & $C_5$ & $C_3$ & $C_2$ & $C_2$ & $C_3$ & $C_3$ & $C_3$ & $C_3$ \\
$ x^5 - 11x^3 + 22x - 11 $         & -   & $C_5$ & $C_5$ & $C_5$ & $C_5$ & $C_1$ & $C_5$ & $C_5$ & $C_5$ & $C_5$ & $C_1$ & $C_5$ \\
$ x^6 + x + 1 $                    & $C_6$ & $C_6$ & $C_4$ & $C_4$ & $C_6$ & $C_5$ & $C_4$ & $C_4$ & $C_6$ & $C_5$ & $C_4$ & $C_3$ \\
$ x^7 - 2x^6 + 2x + 2 $            & $C_5$ & $C_7$ & $C_4$ & $C_7$ & $C_4$ & $C_5$ & $C_7$ & $C_3$ & $C_4$ & $C_7$ & $C_5$ & $C_4$ \\
$ x^7 + x^4 - 2x^2 + 8x + 4 $      & $C_2$ & $C_2$ & $C_6$ & -   & $C_2$ & $C_2$ & $C_2$ & $C_2$ & $C_6$ & $C_2$ & $C_2$ & $C_2$ \\
$ x^7 + x^5 - 4x^4 - x^3 + 5x + 1$ & $C_7$ & $C_7$ & $C_7$ & $C_2$ & $C_2$ & $C_2$ & $C_2$ & $C_7$ & $C_2$ & $C_7$ & $C_7$ & $C_2$ \\ \hline
\end{tabular}
\caption{ Decomposition groups of various polynomials for primes between $43$ and $97$}
\end{center}
\end{table}

\section{Question 4}
\subsection{$x^2+x+41$}
Modulo $2$ this polynomial has Galois Group $C_2$. Hence over $\mathbb{Q}$ it also has Galois group $C_2$ since this is the biggest group a degree 2 polynomial can have.
\subsection{$x^3+2x+1$    }
We can see over $\mathbb{Q}$ this polynomial has a Galois group which contains a $2$ and $3$.  Hence it contains a $2,\deg f-1=2$ and $\deg f$ cycle and so has $S_3$ as it's Galois group.



\subsection{$x^3+x^2-2x-1$    }
We can see over $\mathbb{Q}$ that this polynomial must contain a $3$ cycle  It maximally can have size $6$ so either the group is $D_6$ or $C_6$ or $C_3$.


\subsection{$x^4-2x^2+4$  }

We know that it's Galois group over $\mathbb{Q}$ contains a double transposition  Hence it's Galois group over $\mathbb{Q}$ must be a subgroup of $S_4$ which contains $C_2$ as a subgroup.  Further we know this inclusion is strict since if it's $\text{Gal} (f)$ over $\mathbb{Q}$ was $C_2$ then it would be a reducible polynomial as all degree 2 extensions are quadratic.  Then since $x^4-2x^2+4$ is irreducible, the degree of the extension can't be two. 

\subsection{$x^4-x^3-4x+16$    }

We know that it's Galois group over $\mathbb{Q}$ must be a subgroup of $S_4$ which contains a $4$-cycle,  a double transposition and single transposition.  

\subsection{$ x^4 - 2x^3 + 5x + 5$     }
We can see that $\text{Gal} (f)$ over $\mathbb{Q}$ must contain $C_3,C_4$ as a subgroup,  further we know it must contain a $2$-cycle since it's Galois group over $\mathbb{F}_{43}$ is generated by such a cycle.  Hence $\text{Gal} (f)$ over $\mathbb{Q}$ contains a $\deg f,\deg f-1$ cycle and a transposition and so must $S_4$. 



\subsection{$x^4 + 7x^2 + 6x + 7$    }

First note this polynomial is reducible,  indeed decomposing it into irreducible factors gives
\[x^4 + 7x^2 + 6x + 7=(x^2-x+7)(x^2+x+1).  \]
Further note over $\mathbb{F}_{17}$ it's Galois group is generated by a double transposition and so it's Galois group over $\mathbb{Q}$ must contain a double transposition.  But it's the orbits of the action of $\text{Gal}(f)$ on it's roots correspond to it's irreducible factors so we must in fact have that it's Galois group is $C_2\times C_2$.


\subsection{$x^4 + 3x^3 - 6x^2 - 9x + 7$    }

First note this polynomial is reducible,  indeed decomposing it into irreducible factors gives
\[x^4 + 3x^3 - 6x^2 - 9x + 7=(x^2+x-1)(x^2+2x-7).  \]
Similarly to the previous polynomial we have the Galois group of this polynomial over $\mathbb{Q}$ must be $C_2\times C_2$.

\subsection{$ x^5 + 36$   }
We can see that over $\mathbb{Q}$,  the Galois group of this polynomial contains a double transposition,  a $4$ cycle and a $5$ cycle.  Hence the Galois group of the polynomial must be a subgroup of $S_5$ which contains $C_5,C_4$ as subgroup.


\subsection{$ x^5 - 5x + 3$    }
First note this polynomial is reducible,  indeed decomposing it into irreducible factors gives
\[x^5 - 5x + 3=(x^2+x-1)(x^3-x^2+2x-3).    \]
Then it's Galois group over $\mathbb{Q}$ contains $C_6$ as a subgroup. Hence $Gal(f)$ over $\mathbb{Q}$ is either $C_6,C_2\times D_{12}$.



\subsection{$ x^5 + x^3 - 3x^2 + 3$    }
We can see that over $\mathbb{Q}$,  the Galois group of this polynomial contains $C_2,C_3,C_5$ as subgroups.  Hence the Galois group of this polynomial over $\mathbb{Q}$ is a subgroup of $S_5$ containing a $3,5$ cycle and a double transposition.  Since the finite fields over which this polynomial has $C_2$ as it's Galois group is generated by a double transposition we can't be sure if $\text{Gal}(f)$ is $S_5$, which would be the case if $\text{Gal}(f)$ over $\mathbb{Q}$ contains a single transposition.
\subsection{$ x^5 - 11x^3 + 22x - 11 $    }
We can see that over $\mathbb{Q}$ $\text{Gal}(f)$ contains a $C_5$ a subgroup.  Hence $\text{Gal} (f)$ over $\mathbb{Q}$ is a subgroup of $S_5$ containing $C_5$


\subsection{$ x^6 + x + 1 $   }
We can see that $\text{Gal} (f)$ over $\mathbb{Q}$ is a subgroup of $S_6$ containing $C_4,C_5,C_6$ as subgroups. 


\subsection{$ x^7 - 2x^6 + 2x + 2 $    }

We can see that $\text{Gal} (f)$ over $\mathbb{Q}$ is a subgroup of $S_6$ containing $C_4,C_5,C_6,C_7$ as subgroups. 
\subsection{$ x^7 + x^4 - 2x^2 + 8x + 4 $   }
First note this polynomial is reducible,  indeed decomposing it into irreducible factors gives
\[x^7 + x^4 - 2x^2 + 8x + 4=(x^3+2x+1)(x^4-2x^2+4).    \]
Then it's Galois group over $\mathbb{Q}$ contains $C_6$ as a subgroup.  Note as previously discussed $\text{Gal}_\mathbb{Q}(x^3+2x+1)=D_6$ and $C_2< \text{Gal}_\mathbb{Q}(x^4-2x^2+4)$.  Then certainly $\text{Gal} (f)$ over $\mathbb{Q}$ must contain a copy of $D_6$ and $C_6$.


\subsection{$ x^7 + x^5 - 4x^4 - x^3 + 5x + 1$  }
We can see that $\text{Gal} (f)$ over $\mathbb{Q}$ must contain $C_2$ and $C_7$ as a subgroup.

\newpage
\subsection{Frequency of Cycle Types}

I conjecture that the frequency of cycle types that occurs in the Galois group of a polynomial over $\mathbb{F}_p$ for $p$ primes between $1$ and $N$ is roughly fixed as $N$ gets large.  Below I've included two tables,  both giving the frequency of cycle types for polynomials over $F_p$ between $1$ and $N$.  The first table gives this for $N=100$ and the second for $N=31$.  The data given supports my hypothesis with exception of $f(x)=x^2+x+41$.  The ratio of a $1$ cycle to a $2$ cycle occurring for $N=1000$ is $0.43:0.57$ which does in fact support my conjecture.



\begin{table}[hp]
\begin{center}
\begin{tabular}{|c|ccccccc|}
\hline
\multirow{2}{*}{Polynomial}                                   & \multicolumn{7}{c|}{Cycle Type Frequency}              \\ \cline{2-8} 
                                                     & 1    & 2    & 3    & 4    & 5    & 6    & 7    \\ \hline
$x^2+x+41$                                           & 0.32 & 0.68 & 0    & 0    & 0    & 0    & 0    \\
$x^3+2x+1$                                           & 0.08 & 0.58 & 0.33 & 0    & 0    & 0    & 0    \\
$x^3+x^2-2x-1$                                       & 0.29 & 0    & 0.71 & 0    & 0    & 0    & 0    \\
$x^4-2x^2+4$                                         & 0.22 & 0.78 & 0    & 0    & 0    & 0    & 0    \\
$x^4-x^3-4x+16$                                      & 0.05 & 0.64 & 0    & 0.32 & 0    & 0    & 0    \\
$ x^4 - 2x^3 + 5x + 5$                               & 0    & 0.14 & 0.5  & 0.36 & 0    & 0    & 0    \\
$x^4 + 7x^2 + 6x + 7$             & 0.45 & 0.55 & 0    & 0    & 0    & 0    & 0    \\
$ x^4 + 3x^3 - 6x^2 - 9x + 7$    & 0.18 & 0.82 & 0    & 0    & 0    & 0    & 0    \\
$ x^5 + 36$                                          & 0.05 & 0.23 & 0    & 0.55 & 0.18 & 0    & 0    \\
$ x^5 - 5x + 3$              & 0    & 0.08 & 0.08 & 0.29 & 0.25 & 0.29 & 0    \\
$ x^5 + x^3 - 3x^2 + 3$                              & 0    & 0.18 & 0.45 & 0    & 0.36 & 0    & 0    \\
$ x^5 - 11x^3 + 22x - 11 $                           & 0.13 & 0    & 0    & 0    & 0.87 & 0    & 0    \\
$ x^6 + x + 1 $                                      & 0    & 0    & 0.12 & 0.28 & 0.12 & 0.48 & 0    \\
$ x^7 - 2x^6 + 2x + 2 $                              & 0    & 0    & 0.05 & 0.27 & 0.18 & 0.05 & 0.45 \\
$ x^7 + x^4 - 2x^2 + 8x + 4 $ & 0    & 0.68 & 0.05 & 0    & 0    & 0.27 & 0    \\
$ x^7 + x^5 - 4x^4 - x^3 + 5x + 1$                   & 0    & 0.42 & 0    & 0    & 0    & 0    & 0.58 \\ \hline
\end{tabular}
\caption{Frequency of Cycle Types of Galois Groups of Polynomials Over Primes}
\end{center}
\end{table}


\begin{table}[hp]
\begin{center}
\begin{tabular}{|c|lllllll|}
\hline
\multirow{2}{*}{Polynomial}                                   & \multicolumn{7}{c|}{Cycle Type Frequency}                                                                                                                                      \\ \cline{2-8} 
                                                     & \multicolumn{1}{c}{1} & \multicolumn{1}{c}{2} & \multicolumn{1}{c}{3} & \multicolumn{1}{c}{4} & \multicolumn{1}{c}{5} & \multicolumn{1}{c}{6} & \multicolumn{1}{c|}{7} \\ \hline
$x^2+x+41$                                           & 0                     & 1                     & 0                     & 0                     & 0                     & 0                     & 0                      \\
$x^3+2x+1$                                           & 0.09                  & 0.45                  & 0.45                  & 0                     & 0                     & 0                     & 0                      \\
$x^3+x^2-2x-1$                                       & 0.2                   & 0                     & 0.8                   & 0                     & 0                     & 0                     & 0                      \\
$x^4-2x^2+4$                                         & 0.11                  & 0.89                  & 0                     & 0                     & 0                     & 0                     & 0                      \\
$x^4-x^3-4x+16$                                      & 0                     & 0.63                  & 0                     & 0.38                  & 0                     & 0                     & 0                      \\
$ x^4 - 2x^3 + 5x + 5$                               & 0                     & 0                     & 0.56                  & 0.44                  & 0                     & 0                     & 0                      \\
$x^4 + 7x^2 + 6x + 7$             & 0.38                  & 0.63                  & 0                     & 0                     & 0                     & 0                     & 0                      \\
$ x^4 + 3x^3 - 6x^2 - 9x + 7$    & 0.11                  & 0.89                  & 0                     & 0                     & 0                     & 0                     & 0                      \\
$ x^5 + 36$                                          & 0.13                  & 0.25                  & 0                     & 0.5                   & 0.13                  & 0                     & 0                      \\
$ x^5 - 5x + 3$              & 0                     & 0.1                   & 0.1                   & 0.3                   & 0.1                   & 0.4                   & 0                      \\
$ x^5 + x^3 - 3x^2 + 3$                              & 0                     & 0.22                  & 0.22                  & 0                     & 0.56                  & 0                     & 0                      \\
$ x^5 - 11x^3 + 22x - 11 $                           & 0.1                   & 0                     & 0                     & 0                     & 0.9                   & 0                     & 0                      \\
$ x^6 + x + 1 $                                      & 0                     & 0                     & 0.18                  & 0.09                  & 0.09                  & 0.64                  & 0                      \\
$ x^7 - 2x^6 + 2x + 2 $                              & 0                     & 0                     & 0                     & 0.25                  & 0                     & 0.13                  & 0.63                   \\
$ x^7 + x^4 - 2x^2 + 8x + 4 $ & 0                     & 0.56                  & 0.11                  & 0                     & 0                     & 0.33                  & 0                      \\
$ x^7 + x^5 - 4x^4 - x^3 + 5x + 1$                   & 0                     & 0.4                   & 0                     & 0                     & 0                     & 0                     & 0.6                    \\ \hline
\end{tabular}
\caption{Frequency of Cycle Types of Galois Groups of Polynomials Over Primes between 2 and 31}
\end{center}
\end{table}
\section{Code}
\subsection{Question 1}
\lstinputlisting[style=Matlab-editor]{poldiv.m}
\lstinputlisting[style=Matlab-editor]{polgcd.m}
\subsection{Question 2}
\lstinputlisting[style=Matlab-editor]{decompcalc4.m}


\end{document}