\documentclass{article}
\usepackage{float}
\usepackage{enumitem}
\usepackage{hyperref}
\usepackage[margin=0.5in]{geometry}
\usepackage[framed, numbered]{matlab-prettifier}
\usepackage{amsfonts}
\usepackage{amsmath}
\hypersetup{
    colorlinks,
    citecolor=black,
    filecolor=black,
    linkcolor=black,
    urlcolor=black
}
\usepackage{graphicx}
\graphicspath{ {./Images/} }
\begin{document}
\title{Primality Testing}
\maketitle
\tableofcontents
\newpage
\section*{Preface}
This is my CATAM project, 15.1 for part II. The code for each question can be found in section $5$.
\newpage
\section{Trial Division}
\subsection{Question 1}
Below are the primes in the intervals $[188000,188200]$ and $[10^9,10^9+200]$. 
\begin{table}[hp]
\begin{center}
\begin{tabular}{|l|}
\hline
188011 \\ \hline
188017 \\ \hline
188021 \\ \hline
188029 \\ \hline
188107 \\ \hline
188137 \\ \hline
188143 \\ \hline
188147 \\ \hline
188159 \\ \hline
188171 \\ \hline
188179 \\ \hline
188189 \\ \hline
188197 \\ \hline
\end{tabular}
\end{center}
\caption{Primes between $188000$ and $188200$}
\end{table}
\begin{table}[hp]
\begin{center}
\begin{tabular}{|l|}
\hline
1000000007 \\ \hline
1000000009 \\ \hline
1000000021 \\ \hline
1000000033 \\ \hline
1000000087 \\ \hline
1000000093 \\ \hline
1000000097 \\ \hline
1000000103 \\ \hline
1000000123 \\ \hline
1000000181 \\ \hline
\end{tabular}
\end{center}
\caption{Primes between $10^9$ and $10^9+200$}
\end{table}
\section{The Fermat Test}
\subsection{Question 2}
Below are tables of the pseudo primes in the intervals $[188000,188200]$ in base $2$ to base $13$.  We don't include a table for pseudo primes in the interval $[10^9,10^9+200]$ since there are none. Note for we exclude $a$ for which there are no pseudo primes base $a$ in the interval. 




\begin{table}[hp]
\begin{center}
\begin{tabular}{|c|c|c|c|c|c|c|c|c|}
\hline
\multicolumn{9}{|c|}{\text{Base}}                            \\ \hline
2      & 3      & 4      & 5      & 7      & 8      & 	9	  & 10     & 12      \\ \hline
188057 & 188191 & 188057 & 188113  & 188191 & 188057 & 188191 & 188191 & 188191   \\ \hline
-	   & 	-    & 188191 &    -     &    -    & 	-	 & 	-	  & -  & 	-	   \\ \hline
\end{tabular}
\end{center}
\caption{Pseudo primes between $188000$ and $188200$ in base 2 to base 13}
\end{table}


The complexity of the algorithm is $O(n^2)$. This is because all the loops but 1 are not nested within one another so these are all $O(n)$. The only exception to this is when for each $i^{\text{th}}$ digit of the number $-1$ (in binary) we're testing is $1$ or $0$ and then based on this decide whether or not to include $a^{2^{i-1}}$ in the final product when calculating $a^{p-1}$. Since we do this for each digit this has complexity $O(n^2)$ and hence the complexity of the algorithm is $O(n^2)$.

\newpage
\subsection{Question 3}
There are $43$ absolute pseudo primes and $245$ pseudo primes base $2$. in the interval $[2,10^6]$. This tells us that the vast majority of composite numbers in this interval fail the Fermat test to this base.  Further for the remaining composite numbers,  $151$ failed the Fermat test for the base $3$.  On average for (non absolute pseudo prime) composite numbers which passed the Fermat test base 2, it took $2.74$ bases for them to fail the Fermat test and maximally took $10$ bases for any such number to fail the Fermat test.  In short the probability of a composite number being an absolute pseudo prime is very low in this interval,  and the number of bases required to show such is also quite low.
\section{The Euler Test}
\subsection{Question 4}
There are no absolute Euler pseudo primes, this is due to the Solovay-Strassen Theorem which states for any odd $n>2$, $n$ is prime if and only if for all integers $a$ s.t $(a,n)=1$ we have
\[ \left(\frac{a}{n}\right)=a^{\frac{n-1}2}\pmod n.  \]
The program found no absolute Euler pseudo primes in the interval $[2,10^6]$ and so agrees with this result.\\\\\

There were 114 composite numbers in $[2,10^6]$ which passed the Euler test for $a=2$ and required further checking,  hence the chance of a composite number passing the Euler test for $a=2$ is lower than that of the Fermat test in this interval.  Interestingly however, the average number of bases needed for a composite number to fail the Euler test, if it passed for $a=2$ was higher than that of the Fermat test. On average such numbers needed to be checked with $3.7$ different bases and maximally took $28$ bases for it to fail the Euler test.
\section{The Strong Test}
\subsection{Question 5}
It is again a fact of number theory that there are no absolute strong pseudo primes.  The program found no absolute strong pseudo primes in the interval $[2,10^6]$ and so agrees with this result.\\\\

This was by far the most efficient and quickest prime tester of the three.  There were only $46$ composite numbers which passed the test for $a=2$ and all of these failed the test for $a=3$.  This is much better than the previous three program.





\section{Code}
\subsection{Question 1}
\lstinputlisting[style=Matlab-editor]{Q1.m}
\subsection{Question 2}
\lstinputlisting[style=Matlab-editor]{Q2.m}
\subsection{Question 3}
\lstinputlisting[style=Matlab-editor]{Q3.m}
\subsection{Question 4}
\lstinputlisting[style=Matlab-editor]{Q4.m}
\subsection{Question 5}
\lstinputlisting[style=Matlab-editor]{Q5.m}











\end{document}